%%%%%%%%%%%%%%%%%%%%%%%%%%%%%%%%%%%%%%%%%
% Beamer Presentation
% LaTeX Template
% Version 1.0 (10/11/12)
%
% This template has been downloaded from:
% http://www.LaTeXTemplates.com
%
% License:
% CC BY-NC-SA 3.0 (http://creativecommons.org/licenses/by-nc-sa/3.0/)
%
%%%%%%%%%%%%%%%%%%%%%%%%%%%%%%%%%%%%%%%%%

%----------------------------------------------------------------------------------------
%	PACKAGES AND THEMES
%----------------------------------------------------------------------------------------

\documentclass{beamer}

\mode<presentation> {

% The Beamer class comes with a number of default slide themes
% which change the colors and layouts of slides. Below this is a list
% of all the themes, uncomment each in turn to see what they look like.

%\usetheme{default}
%\usetheme{AnnArbor}
%\usetheme{Antibes}
%\usetheme{Bergen}
%\usetheme{Berkeley}
%\usetheme{Berlin}
%\usetheme{Boadilla}
%\usetheme{CambridgeUS}
%\usetheme{Copenhagen}
%\usetheme{Darmstadt}
%\usetheme{Dresden}
%\usetheme{Frankfurt}
%\usetheme{Goettingen}
%\usetheme{Hannover}
%\usetheme{Ilmenau}
%\usetheme{JuanLesPins}
%\usetheme{Luebeck}
%\usetheme{Madrid}
%\usetheme{Malmoe}
%\usetheme{Marburg}
%\usetheme{Montpellier}
%\usetheme{PaloAlto}
\usetheme{Pittsburgh}
%\usetheme{Rochester}
%\usetheme{Singapore}
%\usetheme{Szeged}
%\usetheme{Warsaw}

% As well as themes, the Beamer class has a number of color themes
% for any slide theme. Uncomment each of these in turn to see how it
% changes the colors of your current slide theme.

%\usecolortheme{albatross}
%\usecolortheme{beaver}
%\usecolortheme{beetle}
%\usecolortheme{crane}
%\usecolortheme{dolphin}
%\usecolortheme{dove}
%\usecolortheme{fly}
%\usecolortheme{lily}
%\usecolortheme{orchid}
%\usecolortheme{rose}
%\usecolortheme{seagull}
%\usecolortheme{seahorse}
%\usecolortheme{whale}
%\usecolortheme{wolverine}

%\setbeamertemplate{footline} % To remove the footer line in all slides uncomment this line
%\setbeamertemplate{footline}[page number] % To replace the footer line in all slides with a simple slide count uncomment this line

%\setbeamertemplate{navigation symbols}{} % To remove the navigation symbols from the bottom of all slides uncomment this line
}

\usepackage{graphicx} % Allows including images
\usepackage{booktabs} % Allows the use of \toprule, \midrule and \bottomrule in tables

%----------------------------------------------------------------------------------------
%	TITLE PAGE
%----------------------------------------------------------------------------------------

\title[Short title]{Parallel Dijkstra Algorithm} % The short title appears at the bottom of every slide, the full title is only on the title page

\author{Yesheng Ma\\Ke Chang} % Your name
\institute[UCLA] % Your institution as it will appear on the bottom of every slide, may be shorthand to save space
{

}
\date{\today} % Date, can be changed to a custom date

\begin{document}

\begin{frame}
\titlepage % Print the title page as the first slide
\end{frame}

\begin{frame}
\frametitle{Overview} % Table of contents slide, comment this block out to remove it
\tableofcontents % Throughout your presentation, if you choose to use \section{} and \subsection{} commands, these will automatically be printed on this slide as an overview of your presentation
\end{frame}

%----------------------------------------------------------------------------------------
%	PRESENTATION SLIDES
%----------------------------------------------------------------------------------------

%------------------------------------------------
\section{Introduction} 


\begin{frame}
\frametitle{Parallel Dijkstra Algorithm}
\begin{itemize}
	\item Similar to serial version
	\item Exploit parallelism in find the vertex with least distance
	\item The key idea is just \textbf{map} and \textbf{reduce}
\end{itemize}
\end{frame}

\section{Environment}
\begin{frame}
\frametitle{Build and Test}
\begin{itemize}
\item Use GNUMakefile to relief you from repeating\\
		\qquad \texttt{make}, \texttt{make clean} etc
\item Write Bash scripts with command line arguments to test:
\begin{itemize}
	\item in batch
	\item with specified argument
\end{itemize}
 
\end{itemize}
\end{frame}

\section{Implementation}
\begin{frame}{MPI API}
	\begin{itemize}[<+->]
		\item \texttt{MPI\_Init}: initialize MPI context
		\item \texttt{MPI\_Comm\_size}: get size of MPI communication space
		\item \texttt{MPI\_Comm\_rank}: get the rank of current thread, i.e. \texttt{my\_rank}
		\item \texttt{MPI\_Allreduce}: similar to reduce in FP(list$\rightarrow$reduced value)
		\item \texttt{MPI\_Gather}: gather small arrays to form a large one
	\end{itemize}
\end{frame}

\begin{frame}[fragile]{Core Implementation}
\begin{itemize}
\item Encapsulate parallel Dijkstra algorithm
\begin{itemize}
	\item[1.] Pass more arguments to \texttt{Dijkstra} function
	\item[2.] But reusability and modularity gained
\end{itemize}

\end{itemize}

\centering
\begin{verbatim}
void Dijkstra(int loc_mat[], int loc_dist[], 
           int loc_pred[], int loc_n, int my_rank, int n);
\end{verbatim}
\end{frame}

\begin{frame}[fragile]{Impl Cont'd}
The implementation itself is quite easy:
\begin{enumerate}
	\item Initialization: \verb|loc_dist, loc_known, loc_pred|
	\item Do n-1 times iteration:
	\quad \begin{enumerate}
		\item \verb|Find_min_loc_dist| and store to \verb|my_min|
		\item do Allreduce to get \verb|glbl_min|
		\item for all unknown vertices, update if possible
	\end{enumerate}
	\item Algorithm finished, print necessary message.
\end{enumerate}
\end{frame}


\begin{frame}{Potential Risks}
Actually, the graph is far from real-world complex networks.
\begin{itemize}
\item May have wired topological structures $\Rightarrow$ other algorithms
\item May have large edge weights $\Rightarrow$ introduce BigInt like Java
\end{itemize}
\end{frame}






%------------------------------------------------

\begin{frame}
\Huge{\centerline{The End}}
\end{frame}

%----------------------------------------------------------------------------------------

\end{document} 